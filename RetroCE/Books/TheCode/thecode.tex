% Created 2019-01-05 Sat 11:54
% Intended LaTeX compiler: pdflatex
\documentclass[11pt]{article}
\usepackage[utf8]{inputenc}
\usepackage[T1]{fontenc}
\usepackage{graphicx}
\usepackage{grffile}
\usepackage{longtable}
\usepackage{wrapfig}
\usepackage{rotating}
\usepackage[normalem]{ulem}
\usepackage{amsmath}
\usepackage{textcomp}
\usepackage{amssymb}
\usepackage{capt-of}
\usepackage{hyperref}
\setcounter{secnumdepth}{2}
\author{Hoyoul Park}
\date{\today}
\title{The Code}
\hypersetup{
 pdfauthor={Hoyoul Park},
 pdftitle={The Code},
 pdfkeywords={org-mode, export, html, theme, style, css, js, bigblow},
 pdfsubject={Org-HTML export made simple.},
 pdfcreator={Emacs 26.1 (Org mode 9.1.14)}, 
 pdflang={English}}
\begin{document}

\maketitle
\tableofcontents

\begin{itemize}
\item if you want go back?  \href{http://www.holytofrege.space/index.html}{home}
\end{itemize}
\section{Preface}
\label{sec:orge604fa1}
\subsection{저자}
\label{sec:orgbed19d4}
찰스 페졸드(Charles Petzold)
\begin{center}
\begin{center}
\includegraphics[width=.9\linewidth]{./img/charles.jpg}
\end{center}   
\end{center}
\subsection{저자의 말}
\label{sec:orga3006b7}
\begin{quote}
10년 정도 고민해서 만든 책. 사람들에게 이책은 컴퓨터의 작동원리를 설명하는 책이라고 함. 사람들은 
\end{quote}
\subsection{영어 단어}
\label{sec:org78619ce}

\section{chapter1. Best Friends}
\label{sec:org37677f2}
\begin{note}
\begin{quote}
 10살짜리 애들이 나온다. 그들은 밤이 되면 서로 얘기하고 싶지만, 통행금지 시간에
걸려 대화를 할 수 없다. 서로간의 대화\ldots{}그것은 인간의 본능이다.라고 petzold는
말한다.
그럼 어떻게 10살짜리들은 대화를 하려고 할까? 불이꺼진 밤에\ldots{}첫번째 시도는 전화를
하는 것이다. 그러나 전화는 다른 가족이 쉽게 눈치챈다. 소리가 나기 때문이다. 그러면
컴퓨터로 대화하는 건 어떨까? 나우누리같은거\ldots{}그런데 컴퓨터는 10살짜리 꼬마방에
있을 수가 없다.

Flashlight다. 후레쉬는 원래 밤에 공부할 학생을 위해 만들어진 것이기 때문에 10살
학생들이라면 누구나 갖고 있다. 10살짜리 학생들은 서로 대화하기 위해서 flashlight
를 사용하기로 한다.

창문에 서서 flashlight로 공중에 글자를 그린다. 그리고 조금 쉬었다가 또 그린다. 그런데
그렇게 그린 문자는 정확하지 않다. 그리고 팔이 너무 아프다. 다른 방법을 찾는다. 
영화에서 보면, 먼곳에 떨어진 사람끼리 flash를 깜빡이면서 서로간의 대화를 하는 것을
본적이 있을 것이다. 그 방법을 이용하기로 한다. A는 1번깜빡임, B는 2번,..Z는 26번..이렇게
하면 BAD라는 문자는 2,1,4번으로 표시할 수 있다. 이렇게 하면 더이상 공중에 팔을 휘둘려
문자를 나타낼 필요는 없다. 그런데 "How are you?"라는 문장을 나타낸다고 생각해보자. 
131번이나 깜빡여야 한다.?는 또 어떻게 나타낼 것인가?

이런 고민을 했던 또다른 사람이 있다. 모르스다. 그리고 그가 발명한 Morse Code다.
Morse code는 깜빡임을 2개로 나눴다. 긴 깜박임, 짧은 깜빡임. 
\end{quote}
\end{note}
\section{chapter2}
\label{sec:orgc2c669b}
\section{chapter3}
\label{sec:org9164a16}
\end{document}
